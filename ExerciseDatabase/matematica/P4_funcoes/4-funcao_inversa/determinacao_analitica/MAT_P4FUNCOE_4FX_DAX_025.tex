% Exercise ID: MAT_P4FUNCOE_4FX_DAX_025
% Created: 2025-11-27
% Difficulty: 2/5

\exercicio{Determine a função inversa de f(x) = 2x + 3. Apresente todos os passos de resolução, incluindo verificação da bijetividade e composição das funções.}

\exercicioDesenvolvimento{
\textbf{Passo 1: Verificar se a função é bijetora}

Para que uma função possua inversa, ela precisa ser bijetora (injetora e sobrejetora).

\textbf{Injetividade:}
Uma função é injetora se f(x₁) = f(x₂) ⇒ x₁ = x₂.

Para f(x) = 2x + 3:
\begin{align*}
f(x_1) &= f(x_2) \\
2x_1 + 3 &= 2x_2 + 3 \\
2x_1 &= 2x_2 \\
x_1 &= x_2
\end{align*}

Portanto, f(x) é injetora.

\textbf{Sobrejetividade:}
Como f(x) = 2x + 3 é uma função linear com coeficiente angular não nulo, seu contradomínio natural é ℝ. Para qualquer y ∈ ℝ, existe x = (y-3)/2 ∈ ℝ tal que f(x) = y. Portanto, f(x) é sobrejetora.

\textbf{Conclusão:} f(x) é bijetora e possui função inversa.

\textbf{Passo 2: Determinar a função inversa}

Aplicamos o algoritmo para encontrar f⁻¹(x):

\begin{enumerate}
    \item Escrevemos y = f(x): \quad y = 2x + 3
    \item Trocamos x por y: \quad x = 2y + 3
    \item Isolamos y: \quad x - 3 = 2y
    \item Resolvemos para y: \quad y = \frac{x - 3}{2}
    \item Substituímos y por f⁻¹(x): \quad f^{-1}(x) = \frac{x - 3}{2}
\end{enumerate}

\textbf{Passo 3: Verificação por composição}

Verificamos se f(f⁻¹(x)) = x e f⁻¹(f(x)) = x:

\begin{align*}
f(f^{-1}(x)) &= 2 \cdot \left(\frac{x - 3}{2}\right) + 3 \\
&= (x - 3) + 3 \\
&= x \quad \checkmark
\end{align*}

\begin{align*}
f^{-1}(f(x)) &= \frac{2x + 3 - 3}{2} \\
&= \frac{2x}{2} \\
&= x \quad \checkmark
\end{align*}

\textbf{Resposta final:} A função inversa é $f^{-1}(x) = \frac{x - 3}{2}$.
}
