% Exercise ID: MAT_A8MODELO_1SX_NFX_013
% Module: Módulo A8 - Modelos Discretos | Concept: Sistemas Numéricos | Type: Números Figurados
% Difficulty: 4/5 (Difícil) | Format: standard
% Tags: binario, decimal, sequencias, hexadecimal, sistemas_numericos, numeros_pentagonais, representacao_visual, conversao_bases, regra_sucessao
% Author: Professor Diogo | Date: 2025-11-26
% Status: active

\exercicio{Analise a sucessão dos números pentagonais representados visualmente abaixo (os valores $P_1$, $P_2$, $P_3$, $P_4$ estão indicados nas figuras):

\begin{center}
\begin{tikzpicture}[scale=0.35]
% P1 = 1
\fill (0,0) circle (0.1);
% Valor $P_1 = 1$ representado na figura

% P2 = 5
\begin{scope}[xshift=3cm]
\foreach \i in {0,72,144,216,288} {
    \fill ({cos(\i)},{sin(\i)}) circle (0.1);
}
% Valor $P_2 = 5$ representado na figura
\end{scope}

% P3 = 12
\begin{scope}[xshift=7cm]
% Camada interior
\foreach \i in {0,72,144,216,288} {
    \fill ({cos(\i)},{sin(\i)}) circle (0.1);
}
% Camada exterior
\foreach \i in {0,72,144,216,288} {
    \fill ({2*cos(\i)},{2*sin(\i)}) circle (0.1);
}
% Valor $P_3 = 12$ representado na figura
\end{scope}

% P4 = 22
\begin{scope}[xshift=13cm]
% Camada interior
\foreach \i in {0,72,144,216,288} {
    \fill ({cos(\i)},{sin(\i)}) circle (0.1);
}
% Camada media
\foreach \i in {0,72,144,216,288} {
    \fill ({2*cos(\i)},{2*sin(\i)}) circle (0.1);
}
% Camada exterior
\foreach \i in {0,72,144,216,288} {
    \fill ({3*cos(\i)},{3*sin(\i)}) circle (0.1);
}
% Valor $P_4 = 22$ representado na figura
\end{scope}
\end{tikzpicture}
\end{center}

\vspace{0.5cm}

\subexercicio{Represente os quatro primeiros termos da sucessao ($P_1, P_2, P_3, P_4$) em sistema binario.}

\subexercicio{Escreva o 5º termo da sucessao ($P_5$) em notacao decimal e hexadecimal.}

\subexercicio{Determine o 7º termo da sucessao ($P_7$) e represente-o em sistema binario.}

\subexercicio{Escreva a regra geral para gerar o proximo termo da sucessao dos numeros pentagonais.}}
