% Exercise ID: MAT_A12OTIMI_ESTU_MRE_002
% Module: Módulo A12 - Otimização | Concept: Estudo da Monotonia
% Type: monotonia_real | Difficulty: 2/5
% Tags: monotonia, gráfico, contexto_real, tanque
% Author: Professor | Date: 2025-11-26

\exercicio{
Num laboratório, o nível de água num tanque foi registado ao longo de uma experiência de 30 minutos. O gráfico mostra como o nível de água variou nesse período.

\begin{center}
\begin{tikzpicture}[scale=1.1]
  % Eixos
  \draw[->] (0,0) -- (7,0) node[right] {Tempo (min)};
  \draw[->] (0,0) -- (0,6) node[above] {Nível de água (L)};
  % Gráfico
  \draw[thick, blue, smooth] plot coordinates {
    (0,1)   % 0 min - 1L
    (2,3)   % 10 min - 3L
    (4,5)   % 20 min - 5L (máximo)
    (6,2)   % 30 min - 2L
  };
  % Marcas no eixo x
  \foreach \x/\label in {0/0, 1/5, 2/10, 3/15, 4/20, 5/25, 6/30}
    \draw (\x,0.1) -- (\x,-0.1) node[below] {\label};
  % Marcas no eixo y
  \foreach \y/\label in {1/1, 2/2, 3/3, 4/4, 5/5}
    \draw (0.1,\y) -- (-0.1,\y) node[left] {\label};
\end{tikzpicture}
\end{center}
}

\subexercicio{Durante que intervalo de tempo o nível de água esteve a aumentar?}

\subexercicio{Durante que intervalo de tempo o nível de água esteve a diminuir?}

\subexercicio{Qual foi o nível máximo de água atingido e em que minuto isso aconteceu?}

\subexercicio{Interpreta o significado do ponto mais alto do gráfico no contexto da experiência.}
