% Exercise ID: MAT_A12OTIMI_ESTU_MPU_002
% Module: Módulo A12 - Otimização | Concept: Estudo da Monotonia
% Type: monotonia_pura | Difficulty: 2/5
% Tags: monotonia, gráfico, ramos, linear
% Author: Professor | Date: 2025-11-26

\exercicio{
O gráfico seguinte representa uma função $g(x)$ definida por ramos no intervalo $[0,3]$:

\begin{center}
\begin{tikzpicture}[scale=1.2]
  \draw[->] (0,0) -- (3.4,0) node[right] {$x$};
  \draw[->] (0,0) -- (0,2.4) node[above] {$g(x)$};
  % ramos
  \draw[thick, red, domain=0:2, samples=50] plot (\x, {abs(\x-1)});
  \draw[thick, red] (2,1) -- (3,1);
  % pontos
  \filldraw[red] (0,1) circle (1pt);
  \filldraw[red] (1,0) circle (1.2pt);
  \filldraw[red] (2,1) circle (1.2pt);
  \filldraw[red] (3,1) circle (1pt);
  % marcas x
  \foreach \x in {0,1,2,3} \draw (\x,0.05) -- (\x,-0.05) node[below] {\x};
  % marcas y
  \foreach \y in {0,1,2} \draw (0.05,\y) -- (-0.05,\y) node[left] {\y};
\end{tikzpicture}
\end{center}
}

% g(x) = |x-1| se 0 <= x < 2; g(x) = 1 se 2 <= x <= 3
% Contexto adaptado: análise de monotonia e extremos de função por ramos valor absoluto/constante


\subexercicio{Indica os intervalos de $x$ onde $g(x)$ é crescente, decrescente e constante.}

\subexercicio{Qual é o valor máximo e o valor mínimo de $g(x)$ no intervalo $[0,5]$? Em que pontos ocorrem?}

\subexercicio{Interpreta o significado dos diferentes ramos do gráfico.}
