% meta:
% id: MAT_P1MODELO_ELE_TAB1_001
% title: "Analisar tabela eleitoral — exemplo A/B/C"
% difficulty: "medium"
% tags: "tabelas,percentagens,maioria,abstenção"
% author: "Auto-generated"
% created_at: 2025-11-20T00:00:00Z
% version: 1
\exercicio{
Tabela:
\begin{center}
\begin{tabular}{lrr}
\toprule
Candidato & Votos & \% (para preencher) \\
\midrule
A & 420 & \\
B & 310 & \\
C & 170 & \\
\midrule
Total votos válidos & 900 & \\
Votos em branco & 25 & \\
Votos nulos & 10 & \\
\bottomrule
\end{tabular}
\end{center}

Perguntas:
\begin{enumerate}
  \item Alguém obteve maioria simples? Resposta e cálculo: \campo[6.0cm]
  \item Alguém obteve maioria absoluta? Resposta e cálculo: \campo[6.0cm]
  \item Calcula as percentagens de cada candidato (com duas casas decimais). \campo[6.0cm]
\end{enumerate}

\vspace{3cm}
}
