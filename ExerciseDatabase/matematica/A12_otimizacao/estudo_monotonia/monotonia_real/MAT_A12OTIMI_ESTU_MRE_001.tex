% Exercise ID: MAT_A12OTIMI_ESTU_MRE_001
% Module: Módulo A12 - Otimização | Concept: Estudo da Monotonia
% Type: monotonia_real | Difficulty: 2/5
% Tags: monotonia, gráfico, contexto_real, temperatura
% Author: Professor | Date: 2025-11-26

\exercicio{
A Maria está a fazer um bolo. A temperatura no interior do bolo foi medida ao longo do tempo, entre as 16h00 e as 16h30. O gráfico seguinte mostra como a temperatura variou durante esse período.

\begin{center}
\begin{tikzpicture}[scale=1.1]
  % Eixos
  \draw[->] (0,0) -- (7,0) node[right] {Tempo (min)};
  \draw[->] (0,0) -- (0,6) node[above] {Temperatura ($^\circ$C)};
  % Gráfico
  \draw[thick, blue, smooth] plot coordinates {
    (0,2)   % 16h00 - 20ºC
    (2,4)   % 16h10 - 60ºC
    (3,5)   % 16h15 - 100ºC (máximo)
    (5,3)   % 16h25 - 60ºC
    (6,2.5) % 16h30 - 40ºC
  };
  % Marcas no eixo x
  \foreach \x/\label in {0/0, 1/5, 2/10, 3/15, 4/20, 5/25, 6/30}
    \draw (\x,0.1) -- (\x,-0.1) node[below] {\label};
  % Marcas no eixo y
  \foreach \y/\label in {2/20, 3/40, 4/60, 5/100}
    \draw (0.1,\y) -- (-0.1,\y) node[left] {\label};
\end{tikzpicture}
\end{center}
}

\subexercicio{Durante que intervalo de tempo a temperatura do bolo esteve a aumentar?}

\subexercicio{Durante que intervalo de tempo a temperatura esteve a diminuir?}

\subexercicio{Qual foi a temperatura máxima atingida e em que minuto isso aconteceu?}

\subexercicio{Interpreta o que significa o ponto mais alto do gráfico no contexto do bolo.}
