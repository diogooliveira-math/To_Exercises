% Exercise ID: MAT_P2ESTATI_1MED_EPO_002
% Module: Módulo P2 - Estatística | Concept: Medições Básicas
% Type: estatistica_poupanca | Difficulty: 2/5
% Tags: media, moda, mediana, poupanca, interpretacao
% Author: Professor | Date: 2025-11-25

\exercicio{
Uma associação de jovens decidiu poupar para uma viagem de finalistas. A tabela seguinte mostra o valor poupado (em euros) por cada membro durante um mês:

\begin{center}
\begin{tabular}{|l|c|}
\hline
\textbf{Membro} & \textbf{Poupança (€)} \\
\hline
Ana & 25 \\
Bruno & 30 \\
Carla & 25 \\
Daniel & 40 \\
Eva & 30 \\
Filipe & 25 \\
Gonçalo & 35 \\
Helena & 30 \\
\hline
\end{tabular}
\end{center}
}

\subexercicio{Calcula a média das poupanças do grupo.}

\subexercicio{Determina a moda. Quantos membros pouparam esse valor?}

\subexercicio{Calcula a mediana das poupanças. Ordena primeiro os valores.}

\subexercicio{Se o objetivo é que cada membro contribua com pelo menos 30€, quantos membros ficaram abaixo desse valor? Usa a mediana para justificar se o grupo está no bom caminho.}
