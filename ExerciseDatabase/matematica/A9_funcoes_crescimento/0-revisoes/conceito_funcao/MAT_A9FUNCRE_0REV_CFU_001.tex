% meta:
% id: MAT_A9FUNCRE_0REV_CFU_001
% module: A9_funcoes_crescimento
% concept: 0-revisoes
% concept_name: Revisões de Crescimento
% tipo: conceito_funcao
% tipo_nome: Conceito de Função
% difficulty: 2
% tags: revisao,funcoes,conceito_funcao
% author: sistema
\exercicio{
Considera a correspondência seguinte entre pessoas e o número de sapatos que calçam:
\[
\begin{tabular}{|c|c|}
\hline
\text{Pessoa} & \text{Número de sapatos que calça} \\
\hline
\text{Ana} & 37 \\
\text{Bruno} & 42 \\
\text{Carla} & 39 \\
\text{David} & 42 \\
\hline
\end{tabular}
\]
\textbf{Pergunta:} Esta correspondência é uma função? Justifica escolhendo a opção correta e explicando por que as outras estão erradas.
\begin{enumerate}
 \item[(A)] Não é uma função, porque o número 42 aparece duas vezes.
 \item[(B)] É uma função, porque cada pessoa está associada a um único número.
 \item[(C)] Não é uma função, porque Bruno e David calçam o mesmo número.
 \item[(D)] Não é uma função, porque há números repetidos na segunda coluna.
\end{enumerate}

}
