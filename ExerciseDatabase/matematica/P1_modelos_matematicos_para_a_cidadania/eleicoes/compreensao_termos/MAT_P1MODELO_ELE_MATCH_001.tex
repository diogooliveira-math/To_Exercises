% meta:
% id: MAT_P1MODELO_ELE_MATCH_001
% title: "Associação: termos eleitorais e definições"
% difficulty: "easy"
% tags: "eleições,definições,compreensão"
% author: "Auto-generated"
% created_at: 2025-11-20T00:00:00Z
% version: 1
\exercicio{
Liga as colunas: (à esquerda termos; à direita definições)

\begin{tabular}{@{}p{0.55\textwidth}p{0.35\textwidth}@{}}
\textbf{Termos} & \textbf{Definições} \\
\midrule
\begin{enumerate}[label=\textbf{(\arabic*)}, leftmargin=*, itemsep=10pt]
  \item Maioria simples
  \item Maioria absoluta
  \item Voto nulo
  \item Voto em branco
  \item Abstenção
\end{enumerate}
&
\begin{enumerate}[label=\textbf{\Alph*}, leftmargin=*, itemsep=8pt]
  \item Exige mais votos que qualquer outro concorrente; não precisa de exceder metade dos votos válidos.
  \item Votos expressos sem contar abstenções; maioria superior a metade dos votos válidos.
  \item Voto inválido por irregularidade na folha (marcações contraditórias, sinais identificadores, etc.).
  \item Voto depositado que não assinala qualquer opção; em muitos sistemas é contabilizado separadamente dos nulos.
  \item Eleitor inscrito que não comparece à mesa de voto no dia da eleição.
\end{enumerate}
\\
\end{tabular}

\vspace{8pt}
\noindent Respostas: 1. \campoLetra \quad 2. \campoLetra \quad 3. \campoLetra \quad 4. \campoLetra \quad 5. \campoLetra

\vspace{3cm}
}
