% Exercise ID: MAT_P4FUNCOE_1GX_JCM_001
% Created: 2025-11-27
% Difficulty: 3/5

\exercicio{Considere as seguintes funções que relacionam o preço $p$ (em euros) com a despesa $E(p)$. Para cada uma, analise a afirmação: "Quando o preço aumenta, a despesa aumenta." Decida se a afirmação é verdadeira para todos os valores de $p$; justifique rigorosamente ou apresente um contraexemplo.\\
\begin{enumerate}[a)]
\item $E(p)=10+2p$.
\item $E(p)=100-3p$.
\item $E(p)=\dfrac{50p}{p+10},\quad p\neq -10$.
\item $E(p)=\begin{cases} p^2,& p\ge 0,\\ -p,& p<0.\end{cases}$
\end{enumerate}

% Metadata file: ExerciseDatabase/matematica/P4_funcoes/1-generalidades_funcoes/juizo_causal/MAT_P4FUNCOE_1GX_JCM_001.json
