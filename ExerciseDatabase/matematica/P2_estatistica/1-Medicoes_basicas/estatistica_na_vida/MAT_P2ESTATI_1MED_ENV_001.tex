% Exercise ID: MAT_P2ESTATI_1MED_ENV_001
% Module: Módulo P2 - Estatística | Concept: Medições Básicas
% Type: estatistica_na_vida | Difficulty: 2/5
% Tags: media, moda, mediana, notas, interpretacao
% Author: Professor | Date: 2025-11-25

\exercicio{
O João registou as suas notas (de 0 a 20 valores) nas primeiras 7 fichas de avaliação do ano letivo:

\begin{center}
\begin{tabular}{|c|c|c|c|c|c|c|}
\hline
\textbf{Ficha 1} & \textbf{Ficha 2} & \textbf{Ficha 3} & \textbf{Ficha 4} & \textbf{Ficha 5} & \textbf{Ficha 6} & \textbf{Ficha 7} \\
\hline
14 & 12 & 15 & 12 & 16 & 14 & 12 \\
\hline
\end{tabular}
\end{center}
}

\subexercicio{Calcula a média das notas do João. Arredonda à unidade.}

\subexercicio{Determina a moda das notas. O que significa este valor para o desempenho do João?}

\subexercicio{Calcula a mediana das notas. Mostra os valores ordenados.}

\subexercicio{Se o João precisar de ter pelo menos 13 valores de média para passar, qual é a nota mínima que precisa na próxima ficha? Justifica.}
