% Exercise ID: MAT_P4FUNCOE_1GEN_AR_001
% Created: 2025-11-27
% Difficulty: 2/5

\exercicio{Classifique cada afirmação como Verdadeira (V) ou Falsa (F) e justifique brevemente.}

\subexercicio{(a) Quando o preço de um bem aumenta, a despesa (gasto total) com esse bem aumenta.}

\subexercicio{(b) Se a quantidade consumida desse bem se mantém fixa, então um aumento do preço implica um aumento da despesa.}

\subexercicio{(c) A relação que associa a cada preço $p$ a despesa total $D(p)=p\cdot q(p)$ é, em toda situação, uma função (ou seja, a cada preço corresponde um único valor de despesa).}

\subexercicio{(d) Se $D(p)=p\cdot q(p)$ é uma função monótona crescente em $p$ (para $p>0$), então obrigatoriamente $q(p)$ é constante e positiva.}

\subexercicio{(e) Se $D(p)=p\cdot q(p)$ é injetora para $p>0$, então $q(p)$ é também uma função injetora.}
