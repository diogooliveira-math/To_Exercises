% Exercise ID: MAT_P2ESTATI_REVI_D_001
% Module: Módulo P2 - Estatística | Concept: Revisões de Crescimento
% Type: desenvolvimento | Difficulty: 2/5
% Tags: revisoes
% Author: Professor | Date: 2025-11-21

\exercicio{
Uma mercearia registou o número de peças de fruta vendidas num dia. A tabela abaixo resume os dados:

\begin{center}
\begin{tabular}{|l|c|}
\hline
\textbf{Fruta} & \textbf{Quantidade} \\
\hline
Maçãs & 45 \\
Bananas & 38 \\
Laranjas & 22 \\
Peras & 15 \\
\hline
\end{tabular}
\end{center}
}

\subexercicio{Indica qual a fruta que representa a maior percentagem das vendas. Justifica a tua resposta com referência aos valores calculados.}

\subexercicio{Calcula quantas peças correspondem a $25\%$ do total de vendas. Indica se deves arredondar o resultado e justifica a tua escolha (arredondamento por defeito, por excesso ou para o inteiro mais próximo).}

\subexercicio{Calcula a percentagem conjunta de Maçãs e Bananas. Essas duas frutas representam mais de $50\%$ do total? Mostra os cálculos e conclui.}
