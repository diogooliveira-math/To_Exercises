% meta:
% id: MAT_A9FUNCRE_1EXP_CPO_107
% module: A9_funcoes_crescimento
% concept: 1-exponenciais
% concept_name: Funções Exponenciais
% tipo: crescimento_populacional
% tipo_nome: Crescimento Populacional
% difficulty: 3
% tags: crescimento_exponencial,modelacao,exponenciais,populacao
% author: sistema
\exercicio{
Numa reserva natural, a população de coelhos evolui segundo: $C(t)=520\times 1,1^t$, onde $t$ (anos) desde 1998.
\textbf{Tarefas:}
\begin{enumerate}
  \item Representa graficamente $C(t)$ (esboço qualitativo).
  
  \begin{center}
  \begin{tikzpicture}[scale=0.8]
    \draw[->] (-0.5,0) -- (10,0) node[right] {$t$};
    \draw[->] (0,-0.5) -- (0,6) node[above] {$C(t)$};
    \foreach \x in {1,2,...,9}
      \draw (\x,0.1) -- (\x,-0.1);
    \foreach \y in {1,2,...,5}
      \draw (0.1,\y) -- (-0.1,\y);
  \end{tikzpicture}
  \end{center}
  
  \item Indica se o gráfico é crescente ou decrescente e interpreta.
  \vspace{3cm}
  \item Escolhe três pontos e explica o significado.
  \vspace{3cm}
  {\renewcommand{\labelenumii}{(\alph{enumii})}
  \begin{enumerate}
    \item Determina \(C(0)\) e \(C(1)\) e interpreta os resultados.
    \vspace{2cm}
    \item Calcula a taxa de crescimento relativa anual e comenta a sua implicação para a população.
    \vspace{2cm}
  \end{enumerate}
  }
  \vspace{3cm}
  \item Cria um problema novo de crescimento exponencial e resolve-o.
  \vspace{3cm}
\end{enumerate}
}
