% Exercise ID: MAT_P4FUNCOE_1GFUN_AFC_001
% Module: MÓDULO P4 - Funções | Concept: Generalidades acerca de Funções
% Difficulty: 3/5 (Médio) | Format: desenvolvimento
% Tags: afirmacoes_causais, funcoes, relacoes_causais, analise_matematica
% Author: Professor | Date: 2025-11-28
% Status: active

\exercicio{Considere P (preço) e D (despesa) relacionadas por uma função D = f(P). Para cada uma das afirmações seguintes: (i) indique se é verdadeira ou falsa; (ii) justifique a sua resposta com argumentos matemáticos; (iii) quando possível, dê um exemplo explícito de função f que torne a afirmação verdadeira e outro que a torne falsa.
\par
\subexercicio{(A) "Quando o preço aumenta, a despesa aumenta."}
\subexercicio{(B) "Se o preço dobra, então a despesa dobra."}
\subexercicio{(C) "Se f for estritamente crescente, então qualquer aumento do preço implica um aumento da despesa."}
\subexercicio{(D) "Se f for constante, então a despesa não varia quando o preço varia."}}