% meta:
% id: MAT_P1MODELO_PERC_CALC_001
% title: "Percentagens: aumentos, descontos e reversão"
% difficulty: "easy"
% tags: "percentagens,desconto,aumento,regra_de_três"
% author: "Auto-generated"
% created_at: 2025-11-20T00:00:00Z
% version: 1
\exercicio{
Instruções: resolve cada item e escreve a resposta nos campos indicados.
\begin{enumerate}[label=\textbf{\arabic*.}, leftmargin=*, itemsep=12pt]
\item O salário base é 870\,€. Vai aumentar 6{,}1\%. Quanto ficará o salário? \\
Resposta: \campoCents

\item O salário base é 1\,245\,€. Aumenta 4{,}75\%. Quanto ficará o salário (arredonda a cêntimos)? \\
Resposta: \campoCents

\item Um produto custa 320\,€. Aplica-se um desconto de 12{,}5\%. Qual é o preço depois do desconto? \\
Resposta: \campoCents

\item O valor passou de 5,00\,€ para 7,50\,€. Qual foi a percentagem do aumento? \\
Resposta: \campo[2.5cm]

\item Um passe mensal custava 60\,€ e passou a custar 51\,€. Qual foi a percentagem de redução? \\
Resposta: \campo[2.5cm]

\item Um salário de 900\,€ aumenta 3\% no mês 1 e mais 2{,}5\% no mês 2. Qual é o salário final (arredonda a cêntimos)? \\
Resposta: \campoCents

\item Depois de um aumento de 20\% um produto custa 180\,€. Quanto custava antes do aumento? \\
Resposta: \campoCents
\end{enumerate}

\vspace{3cm}
}
