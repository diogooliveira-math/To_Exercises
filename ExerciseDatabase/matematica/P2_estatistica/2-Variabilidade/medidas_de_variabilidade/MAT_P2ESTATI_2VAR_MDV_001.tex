% Exercise ID: MAT_P2ESTATI_2VAR_MDV_001
% Module: Módulo P2 - Estatística | Concept: Variabilidade
% Type: medidas_de_variabilidade | Difficulty: 2/5
% Tags: amplitude, desvio_absoluto_medio, variabilidade, poupanca
% Author: Professor | Date: 2025-11-25

\exercicio{
Uma família registou as suas poupanças mensais (em euros) durante 5 meses:

\begin{center}
$\{120, 180, 150, 200, 100\}$
\end{center}
}

\subexercicio{Calcula a amplitude das poupanças. Identifica o valor máximo e o valor mínimo.}

\subexercicio{Calcula a média das poupanças.}

\subexercicio{Calcula o Desvio Absoluto Médio (DAM) usando a fórmula:
\[
\text{DAM} = \frac{\sum_{i=1}^{n} |x_i - \bar{x}|}{n}
\]
Para isso:
\begin{enumerate}
\item Calcula a diferença de cada valor para a média (em módulo).
\item Soma todas as diferenças.
\item Divide pelo número de valores.
\end{enumerate}}

\subexercicio{O que indica um DAM alto ou baixo sobre a consistência das poupanças da família? Interpreta o valor que obtiveste.}
