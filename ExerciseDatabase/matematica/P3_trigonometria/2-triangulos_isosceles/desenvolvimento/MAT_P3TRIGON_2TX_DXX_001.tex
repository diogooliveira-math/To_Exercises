% Exercise ID: MAT_P3TRIGON_2TX_DXX_001
% Created: 2025-11-27
% Difficulty: 2/5

\exercicio{Num triângulo $ABC$ com $AB = AC$ e ângulo em $A = 20^\circ$, encontre os ângulos $B$ e $C$.

\begin{center}
\begin{tikzpicture}[scale=2]
    % Definir pontos
    \coordinate (A) at (0,2);
    \coordinate (B) at (-1.5,0);
    \coordinate (C) at (1.5,0);
    
    % Desenhar triângulo
    \draw[thick] (A) -- (B) -- (C) -- cycle;
    
    % Marcar lados iguais
    \draw[dashed] (A) -- (B);
    \draw[dashed] (A) -- (C);
    \node[above left] at ($(A)!0.5!(B)$) {$a$};
    \node[above right] at ($(A)!0.5!(C)$) {$a$};
    \node[below] at ($(B)!0.5!(C)$) {$b$};
    
    % Marcar ângulos
    \draw[fill=blue!20] (0.5,0) arc (0:20:0.5);
    \node at (0.7,0.2) {$20^\circ$};
    
    % Marcar pontos
    \fill (A) circle (1pt) node[above] {$A$};
    \fill (B) circle (1pt) node[below left] {$B$};
    \fill (C) circle (1pt) node[below right] {$C$};
    
    % Indicar ângulos desconhecidos
    \draw[fill=red!20] (-1.3,0.3) arc (160:180:0.5);
    \draw[fill=red!20] (1.3,0.3) arc (20:0:0.5);
    \node[red] at (-1.5,0.6) {$?$};
    \node[red] at (1.5,0.6) {$?$};
\end{tikzpicture}
\end{center}

\textbf{Instruções de compilação:}
\begin{itemize}
    \item Use \texttt{pdflatex} ou \texttt{xelatex} para compilar
    \item Inclua os pacotes: \texttt{\textbackslash usepackage\{tikz\}} e \texttt{\textbackslash usepackage\{calc\}}
    \item O pacote \texttt{calc} é necessário para as coordenadas intermediárias com \texttt{\$()}
\end{itemize}}
