% meta:
% id: MAT_P1MODELO_ELE_PERC_001
% title: "Calcular votos a partir de percentagens — vários exemplos"
% difficulty: "medium"
% tags: "percentagens,total,votos,conversão"
% author: "Auto-generated"
% created_at: 2025-11-20T00:00:00Z
% version: 1
\exercicio{
Preenche a tabela com os votos calculados a partir das percentagens e do total, e responde:
\begin{enumerate}
\item Considera a seguinte tabela com percentagens de votos. Preenche os votos de cada candidato, sabendo que o total de votos válidos é 1200, com 10 votos em branco e 5 nulos.
\begin{center}
\begin{tabular}{lrr}
\toprule
\textbf{Candidato} & \textbf{\%} & \textbf{Votos (para preencher)}\\
\midrule
A & 38\% & \\
B & 34\% & \\
C & 25\% & \\
D & 3\% & \\
\midrule
\textbf{Total votos válidos} & & 1\,200\\
Votos em branco & & 10\\
Votos nulos & & 5\\
\bottomrule
\end{tabular}
\end{center}
\begin{enumerate}
  \item Calcula o número de votos de cada candidato. \campo[6.0cm]
  \item Alguém obteve maioria absoluta? \campo[6.0cm]
  \item Percentagens de brancos e nulos. \campo[6.0cm]
\end{enumerate}
\item Considera a seguinte tabela com percentagens de votos. Preenche os votos de cada candidato, sabendo que o total de votos válidos é 2000, com 15 votos em branco e 5 nulos.
\begin{center}
\begin{tabular}{lrr}
\toprule
\textbf{Candidato} & \textbf{\%} & \textbf{Votos (para preencher)}\\
\midrule
E & 40\% & \\
F & 30\% & \\
G & 20\% & \\
H & 10\% & \\
\midrule
\textbf{Total votos válidos} & & 2\,000\\
Votos em branco & & 15\\
Votos nulos & & 5\\
\bottomrule
\end{tabular}
\end{center}
\begin{enumerate}
  \item Calcula o número de votos de cada candidato. \campo[6.0cm]
  \item Alguém obteve maioria absoluta? \campo[6.0cm]
  \item Percentagens de brancos e nulos. \campo[6.0cm]
\end{enumerate}
\item Considera a seguinte tabela com percentagens de votos. Preenche os votos de cada candidato, sabendo que o total de votos válidos é 800, com 20 votos em branco e 0 nulos.
\begin{center}
\begin{tabular}{lrr}
\toprule
\textbf{Candidato} & \textbf{\%} & \textbf{Votos (para preencher)}\\
\midrule
I & 45\% & \\
J & 35\% & \\
K & 15\% & \\
L & 5\% & \\
\midrule
\textbf{Total votos válidos} & & 800\\
Votos em branco & & 20\\
Votos nulos & & 0\\
\bottomrule
\end{tabular}
\end{center}
\begin{enumerate}
  \item Calcula o número de votos de cada candidato. \campo[6.0cm]
  \item Alguém obteve maioria absoluta? \campo[6.0cm]
  \item Percentagens de brancos e nulos. \campo[6.0cm]
\end{enumerate}
\end{enumerate}

\vspace{3cm}
}
