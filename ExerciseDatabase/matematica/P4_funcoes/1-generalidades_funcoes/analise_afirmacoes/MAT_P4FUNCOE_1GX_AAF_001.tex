% meta:
% id: MAT_P4FUNCOE_1GX_AAF_001
% title: "Análise de afirmações sobre preço e despesa"
% discipline: "matematica"
% module: "P4_funcoes"
% concept: "1-generalidades_funcoes"
% tipo: "analise_afirmacoes"
% difficulty: 2
% tags: "analise,afirmacoes,preco,despesa"
% author: "OpenAI"
% version: 1

\section{Análise de afirmações}

\exercicio{
Considere a seguinte situação:

"Numa loja, o preço de um produto é representado por $p$ (em euros) e a despesa total dos clientes é dada por $D(p)$.")

Para cada uma das afirmações seguintes, indique se é verdadeira ou falsa, justificando a sua resposta:
\begin{enumerate}
    \item Quando o preço aumenta, a despesa total $D(p)$ aumenta sempre.
    \item Se o preço for zero, a despesa total $D(0)$ é necessariamente zero.
    \item Se a despesa total $D(p)$ diminui quando o preço aumenta, então os clientes estão a comprar menos produtos.
\end{enumerate}
}
