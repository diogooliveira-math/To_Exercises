% meta:
% id: MAT_P1MODELO_ELE_TAB3_001
% title: "Tabelas com abstenção e percentagens — exemplo H/I"
% difficulty: "medium"
% tags: "abstenção,percentagens,tabelas"
% author: "Auto-generated"
% created_at: 2025-11-20T00:00:00Z
% version: 1
\exercicio{
Contém um exemplo com abstenção. Tabela:
\begin{tabular}{lrr}
\toprule
Candidato & Votos & \% \\
\midrule
H & 2\,300 & \\
I & 1\,900 & \\
\midrule
Total votos válidos & 4\,200 & \\
Votos em branco & 150 & \\
Votos nulos & 50 & \\
Eleitores inscritos & 6\,000 & \\
\bottomrule
\end{tabular}

Perguntas:
\begin{enumerate}
  \item Alguém obteve maioria simples? Resposta e cálculo: \campo[6.0cm]
  \item Alguém obteve maioria absoluta? Resposta e cálculo: \campo[6.0cm]
  \item Qual é a taxa de abstenção (em percentagem)? Resposta e cálculo: \campo[6.0cm]
  \item Percentagem de votos em branco e nulos? Resposta e cálculo: \campo[6.0cm]
\end{enumerate}

\vspace{3cm}
}
