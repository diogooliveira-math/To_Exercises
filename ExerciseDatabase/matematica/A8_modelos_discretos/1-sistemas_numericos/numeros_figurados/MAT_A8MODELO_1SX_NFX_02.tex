% Exercise ID: MAT_A8MODELO_1SX_NFX_012
% Module: Módulo A8 - Modelos Discretos | Concept: Sistemas Numéricos | Type: Números Figurados
% Difficulty: 3/5 (Médio) | Format: standard
% Tags: sistemas_numericos, binario, decimal, sequencias, hexadecimal, numeros_quadrados, representacao_visual, conversao_bases, regra_sucessao
% Author: Professor Diogo | Date: 2025-11-26
% Status: active

\exercicio{Observe a sucessão dos números quadrados representados visualmente abaixo (os valores $Q_1$, $Q_2$, $Q_3$, $Q_4$ estão indicados nas figuras):

\begin{center}
\begin{tikzpicture}[scale=0.4]
% Q1 = 1
\foreach \x in {0} {
    \foreach \y in {0} {
        \fill (\x,\y) circle (0.12);
    }
}
% Valor $Q_1 = 1$ representado na figura

% Q2 = 4
\begin{scope}[xshift=3cm]
\foreach \x in {0,1} {
    \foreach \y in {0,1} {
        \fill (\x,\y) circle (0.12);
    }
}
% Valor $Q_2 = 4$ representado na figura
\end{scope}

% Q3 = 9
\begin{scope}[xshift=7cm]
\foreach \x in {0,1,2} {
    \foreach \y in {0,1,2} {
        \fill (\x,\y) circle (0.12);
    }
}
% Valor $Q_3 = 9$ representado na figura
\end{scope}

% Q4 = 16
\begin{scope}[xshift=12cm]
\foreach \x in {0,1,2,3} {
    \foreach \y in {0,1,2,3} {
        \fill (\x,\y) circle (0.12);
    }
}
% Valor $Q_4 = 16$ representado na figura
\end{scope}
\end{tikzpicture}
\end{center}

\vspace{0.5cm}

\subexercicio{Represente os quatro primeiros termos da sucessao ($Q_1, Q_2, Q_3, Q_4$) em sistema hexadecimal.}

\subexercicio{Escreva o 5º termo da sucessao ($Q_5$) em notacao decimal e binaria.}

\subexercicio{Determine o 8º termo da sucessao ($Q_8$) e represente-o em sistema hexadecimal.}

\subexercicio{Escreva a regra geral para gerar o proximo termo da sucessao dos numeros quadrados.}}
