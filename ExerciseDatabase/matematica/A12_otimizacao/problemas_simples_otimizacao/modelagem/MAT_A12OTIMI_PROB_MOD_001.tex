% Exercise ID: MAT_A12OTIMI_PROB_MOD_001
% Module: Módulo A12 - Otimização | Concept: Problemas Simples de Otimização
% Type: modelagem | Difficulty: 2/5
% Tags: modelagem, funcao, custo, interpretacao
% Author: Professor | Date: 2025-11-25

\exercicio{
O custo total $C$ (em euros) de produção de $x$ peças numa fábrica é dado pela função:
\[
C(x) = 5x + 200
\]
onde $x$ representa o número de peças produzidas.
}

\subexercicio{Qual é o custo fixo de produção (custo quando $x = 0$)?}

\subexercicio{Qual é o custo de produzir uma unidade adicional (custo marginal)?}

\subexercicio{Se a fábrica tem um orçamento de 700€, quantas peças pode produzir no máximo? Resolve a equação $C(x) = 700$.}

\subexercicio{Se cada peça é vendida a 12€, a partir de quantas peças a fábrica começa a ter lucro?}
