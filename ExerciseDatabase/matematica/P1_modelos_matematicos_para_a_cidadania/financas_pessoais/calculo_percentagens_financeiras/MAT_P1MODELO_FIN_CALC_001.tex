% meta:
% id: MAT_P1MODELO_FIN_CALC_001
% title: "Percentagens em Finanças Pessoais: juros, impostos e orçamentos"
% difficulty: "easy"
% tags: "percentagens,financas_pessoais,juros,impostos,orcamento"
% author: "Auto-generated"
% created_at: 2025-11-20T11:30:00Z
% version: 1
\exercicio{
Instruções: resolve cada item relacionado a finanças pessoais/familiares e escreve a resposta nos campos indicados.
\begin{enumerate}[label=\textbf{\arabic*.}, leftmargin=*, itemsep=12pt]
\item Um depósito a prazo de 5\,000\,€ rende juros de 3{,}5\% ao ano. Quanto receberá ao final do ano? \\
Resposta: \campoCents

\item O salário mensal é 1\,800\,€. São deduzidos 15\% para impostos. Qual é o salário líquido? \\
Resposta: \campoCents

\item Uma família tem um orçamento mensal de 3\,000\,€. Gasta 25\% em habitação. Quanto gasta em habitação? \\
Resposta: \campoCents

\item Um investimento de 10\,000\,€ teve um retorno de 8\% no primeiro ano. Qual é o valor final do investimento? \\
Resposta: \campoCents

\item Os impostos sobre um rendimento de 25\,000\,€ ao ano são 20\%. Quanto paga em impostos? \\
Resposta: \campoCents

\item Uma conta poupança rende 2\% ao ano. Se depositar 2\,500\,€, quanto terá ao final do ano? \\
Resposta: \campoCents

\item Depois de um aumento de 10\% no salário, este passa a ser 2\,200\,€. Qual era o salário anterior? \\
Resposta: \campoCents
\end{enumerate}

\vspace{3cm}
}
