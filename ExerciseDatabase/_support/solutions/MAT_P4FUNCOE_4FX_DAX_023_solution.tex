% Exercise ID: MAT_P4FUNCOE_4FX_DAX_023
% Solution File
% Created: 2025-11-27
% Difficulty: 2/5

\exercicioDesenvolvimento{Solução: Determinação da função inversa de $f(x) = 2x + 3$}

\textbf{Passo 1: Verificar se a função é injetora}

Uma função é injetora (um-para-um) se $f(a) = f(b) \Rightarrow a = b$.

Seja $f(a) = f(b)$:
$$2a + 3 = 2b + 3$$
$$2a = 2b$$
$$a = b$$

Portanto, $f$ é injetora.

\textbf{Passo 2: Verificar se a função é sobrejetora}

Uma função é sobrejetora (sobre) se para todo $y \in \mathbb{R}$, existe $x \in \mathbb{R}$ tal que $f(x) = y$.

Seja $y \in \mathbb{R}$ qualquer. Queremos encontrar $x$ tal que:
$$2x + 3 = y$$
$$2x = y - 3$$
$$x = \frac{y - 3}{2}$$

Como $\frac{y - 3}{2} \in \mathbb{R}$ para qualquer $y \in \mathbb{R}$, a função é sobrejetora.

\textbf{Conclusão:} Como $f$ é injetora e sobrejetora, ela é \textbf{bijetora} e possui função inversa.

\textbf{Passo 3: Determinar a função inversa}

Para encontrar $f^{-1}(x)$, seguimos o algoritmo padrão:

\begin{enumerate}
    \item Trocar $f(x)$ por $y$:
    $$y = 2x + 3$$
    
    \item Trocar $x$ por $y$ e $y$ por $x$:
    $$x = 2y + 3$$
    
    \item Isolar $y$:
    $$x - 3 = 2y$$
    $$y = \frac{x - 3}{2}$$
    
    \item Substituir $y$ por $f^{-1}(x)$:
    $$f^{-1}(x) = \frac{x - 3}{2}$$
\end{enumerate}

\textbf{Passo 4: Verificação}

Vamos verificar que $f(f^{-1}(x)) = x$ e $f^{-1}(f(x)) = x$:

$$f(f^{-1}(x)) = f\left(\frac{x - 3}{2}\right) = 2 \cdot \frac{x - 3}{2} + 3 = x - 3 + 3 = x \checkmark$$

$$f^{-1}(f(x)) = f^{-1}(2x + 3) = \frac{(2x + 3) - 3}{2} = \frac{2x}{2} = x \checkmark$$

\textbf{Resposta Final:} A função inversa de $f(x) = 2x + 3$ é:
$$\boxed{f^{-1}(x) = \frac{x - 3}{2}}$$