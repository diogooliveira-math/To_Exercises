% meta:
% id: MAT_P1MODELO_ELE_CONC_001
% title: "Verdadeiro/Falso — conceitos eleitorais"
% difficulty: "easy"
% tags: "eleições,maioria,abstenção,votos"
% author: "Auto-generated"
% created_at: 2025-11-20T00:00:00Z
% version: 1
\exercicio{
Instrucções: Para cada afirmação, escreve ``V'' se considerares a afirmação verdadeira ou ``F'' se considerares falsa.

\begin{enumerate}[label=\textbf{(\arabic*)}, leftmargin=*, itemsep=1.0\baselineskip]
  \item A maioria simples exige que um candidato obtenha mais votos do que qualquer outro, independentemente do número total de votos válidos.
  \item A maioria absoluta significa obter mais de metade dos votos válidos.
  \item Votos nulos não entram na contagem dos votos válidos.
  \item A abstenção corresponde a eleitores que se registaram mas não compareceram à votação.
  \item Para uma eleição que exige maioria absoluta, se nenhum candidato obtiver mais de metade dos votos válidos, realiza-se um segundo turno entre os dois mais votados.
  \item Se um eleitor deposita uma folha de voto em que não marca nenhum candidato, esse voto é considerado voto em branco.
\end{enumerate}

\vspace{3cm}
}
